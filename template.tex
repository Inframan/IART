\documentclass[a4paper]{article}

\usepackage[portuguese]{babel}
\usepackage[utf8]{inputenc}
\usepackage{indentfirst}
\usepackage{graphicx}
\usepackage{verbatim}
\usepackage{url}
\usepackage{hyperref}

\bibliographystyle{plain}


\begin{document}

\setlength{\textwidth}{16cm}
\setlength{\textheight}{22cm}

\title{\Huge\textbf{Redes Neuronais para a Predição da Origem Geográfica de Música}\linebreak\linebreak\linebreak
\Large\textbf{Relatório Intercalar}\linebreak\linebreak
\linebreak\linebreak
\includegraphics[scale=0.1]{feup-logo.png}\linebreak\linebreak
\linebreak
\Large{Mestrado Integrado em Engenharia Informática e Computação} \linebreak\linebreak
\Large{Inteligência Artifical}\linebreak
}

\author{\textbf{Grupo E4\_1:}\\ Gabriel Souto - 201208167 \\ José Cardoso - 201202838 \\\linebreak\linebreak \\
 \\ Faculdade de Engenharia da Universidade do Porto \\ Rua Roberto Frias, s\/n, 4200-465 Porto, Portugal }
\pagebreak


\maketitle
\thispagestyle{empty}

%************************************************************************************************

\newpage

%%%%%%%%%%%%%%%%%%%%%%%%%%
\section{Objectivo}

O objetivo deste trabalho consiste na aplicação de Redes Neuronais Artificiais na predição da origem geográfica de música, recorrendo ao algoritmo de \textit{Back-Propagation}.\linebreak
A partir de um conjunto de exemplos, é possível treinar uma Rede Neuronal, para que esta depois possa ser usada na classificação de novos casos.



%%%%%%%%%%%%%%%%%%%%%%%%%%
\section{Descrição}

\subsection{Especificação} Este trabalho irá consistir em várias partes. 
\\Numa primeira fase, iremos analisar os dados fornecidos  pela \textit{UCI Machine Learning Repository} \href{http://tinyurl.com/qcxjzap}{aqui}.

\subsection{Trabalho efetuado} (Pode ser idêntico ao descrito no relatório intercalar.)

\subsection{Resultados esperados e forma de avaliação} Obtenção de uma lista de jogadas possíveis. Exemplo: \textit{valid\_moves(+Board, -ListOfMoves)}.


%%%%%%%%%%%%%%%%%%%%%%%%%%
\section{Conclusões}

Descrever o módulo de interface com o utilizador em modo de texto.
Que conclui deste projecto? Como poderia melhorar o trabalho desenvolvido?

%%%%%%%%%%%%%%%%%%%%%%%%%%
\section{Recursos}

[1] Fang Zhou, Claire Q and Ross. D. King. Predicting the Geographical Origin of Music, ICDM.
\url{http://archive.ics.uci.edu/ml/datasets/Geographical+Original+of+Music}, 2014

\clearpage


\end{document}
